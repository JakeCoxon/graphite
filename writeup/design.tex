\section{Design}

\subsection{Language}
  The language and platform that the program runs on needs to be considered. Since the author's main operating system is Apple Mac OSX and the university's main operating systems are Microsoft Windows and Linux then it is required that the language chosen will run on these three operating systems.

  \subsubsection{The Java Virtual Machine}
    The first obvious choice is the \emph{Java} system by Sun Microsystems in 1991 and now Oracle Corporation since 2009. Java is a cross-platform runtime environment which runs on over 1.1 billion desktop computers worldwide\footnote{http://www.java.com/en/about/}.

    Java achieves cross-platform by using a virtual machine called Java Virtual Machine or JVM. Java programs are written and compiled into Java bytecode which can be subsequently run on any JVM implementation. JVM implementations exist on all major operating systems and Mac OSX bundles a JVM with installation.

    Other than the excellent cross-platform support, the JVM has more advantages. The Java runtime has great built-in graphical user interface library called Swing which can be used to create interfaces that will work on each operating system. This is good for the program created for this project since it provides a unified interface.

    The maturity and widespread use of the JVM leads to many highly used third-party libraries. Many different graph libraries exist that will be considered later.

    The fact that the Java system is used by billions of devides ensures that support for the JVM will continue into the future. This is important for users such as myself that wish to devote time into building an application and want that application to continue to work in many years to come.

    The speed of the JVM has increased over the years due to many optimisation techniques such as Just-in-time compilation and HotSpot which analyses a programs performance and optimised frequently excecuted code paths. Another technique is garbage collection which attempts to reclaim memory that is no longer in use by the program.

    Programs that involve computationally expensive procedures---such as this project---it is beneficial that the program can distribute computations across multiple cores of a computer. This is only possible if the runtime environment supports concurrency. Java has been designed to support concurrency by having built-in constructs for threading. 

    \paragraph{}

    \textbf{Java} is also the name of the main programming language that is supported by the Java runtime. It has a similar style to the popular languages C/C++ but currently lacks in certain modern language features to aid development. The Java Virtual Machine runs Java bytecode which means third-party languages are being developed for it such as Groovy, Clojure and Kotlin. In fact, any language can be compiled to Java bytecode as long as it can be internally expressed as Java bytecode. Some examples include JavaScript as Rhino, Python as Jython and Ruby as JRuby.

    \textbf{Scala} is such a language that has been developed to run on the JVM. Scala first appeared in 2003 and has been a popular choice over Java for a number of reasons.
    Scala has all the positive aspects of running on the JVM as well as improved syntax, type inference, pattern matching, operator overloading. These features help to improve expressiveness.

    Scala also has many functional characteristics such as anonymous functions and closures as first-class citizens, immutable data structures, map/filter/fold methods. These features are a huge benefit over Java especially when implementing mathematical concepts as this project does. For example using higher-order functions makes it easier to describe exactly what needs to be computed while abstracting implementation details away.

    On the other hand, Scala is a relatively new language compared to Java and as such it is slightly `rough around the edges'. The Integrated Development Environment support is not as good as Java and occasionally compile errors are slightly confusing but this is a steady process and these things are improving over time. There is also new syntax to learn which is a potential problem for users.

    In conclusion, using Scala provides improved programming constructs while still employing the benefits of the JVM.

\subsection{External Libraries}
  A code library is a collection of code that one or more people have brought together in a way that other people can use. The use of external libraries in an application can aid development because this code is generally proved to work by use of automated tests. This can cut time of development since it is a section of code that the developer does not need to worry about.

  For this project, it would be beneficial to use a graph library where many graph modification operations are already implemented.

  \subsubsection{Java Universal Network/Graph Framework}
  Java Universal Network/Graph Framework or \emph{JUNG} is a Java library that provides functions for the modeling, analysis and visualisation of graphs and networks. The JUNG architecture supports a variety of graph representations such as directed/undirected, multi-graphs and importantly hypergraphs.

  JUNG provides a highly extensible visualisation framework which enables a developer to show the structure of a graph in a range of different ways with custom layouts, filtering mechanisms and styles.

  The library was designed by a software engineer working at Google and a researcher at Microsoft. It is also partially funded by the National Science Foundation which shows the library is high Unfortunately the last update was in 2011 but it is opensource and available under the Berkeley Software Distribution (BSD) license so this means anybody is free to contribute to it.

  Internally, JUNG uses interfaces to refer to different types of graph. In programming an interface defines what operations the object must have but the implementation can choose how to do it. JUNG supports hypergraph interfaces but only has a concrete implementation for `set hypergraphs' which do not order the tentacles. Fortunately JUNG is designed to be extensible and implementing an sequential hypergraph is trivial. Furthermore, there is no built in support for visualising hypergraphs but again, this can be done fairly easily.

\subsection{File Representation}
